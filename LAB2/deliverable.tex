\documentclass[12]{article}
\usepackage[utf8]{inputenc}
\usepackage{cite}
\author{par4111 \\ Adrià Cabeza, Xavier Lacasa \\ Departament d' Arquitectura de Computadors}
\title{Lab 2: Brief tutorial on OpenMP programming model }
\date{\today \\ 2018 - 19 PRIMAVERA}
\usepackage{graphicx}
\usepackage{subcaption}
\usepackage{pgfplots}
\usepackage{listings}
\usepackage{color}
\usepackage{float}
\definecolor{mauve}{rgb}{0.58,0,0.82}
\definecolor{dkgreen}{rgb}{0,0.6,0}
\lstset{
	frame=tb,language=C,breaklines=true,numbers=none, commentstyle=\color{dkgreen}, stringstyle=\color{mauve}, tabsize=3,   showstringspaces=false,
  columns=flexible, 
}
\begin{document}
\maketitle

\newpage
\tableofcontents
\newpage
\section{Introduction}

In this session we will learn about the OpenMP programming model. 

\section{Parallel regions}
\subsection{hello.c}
\textbf{1. How many times will you see the "Hello world!" message if the program is executed with \textit{./1.hello}?}

We see the message 24 times. This is due to the \textit{\#pragma omp parallel} call before \textit{printf("Hello world! \textbackslash n");}, which makes every available thread execute the \textit{printf}. In Boada 1, they happen to be 24 threads, and so the message is printed 24 times.

\textbf{2. Without changing the program, how to make it to print 4 times the \textit{Hello World!} message?}

By setting the number of threads available to 4, only 4 threads would execute the \textit{printf("Hello world! \textbackslash n");} line and so the message would be displayed only 4 times. 

\subsection{2.hello.c}
\textbf{1. Is the execution of the program correct? (i.e., prints a sequence of \textit{(Thid) Hello (Thid)
world!} being Thid the thread identifier). If not, add a data sharing clause to make it correct?} 

\textbf{2. Are the lines always printed in the same order? Why the messages sometimes appear intermixed?
(Execute several times in order to see this).}

\subsection{how\_many.c}
Assuming the \textit{OMP NUM THREADS} variable is set to 8 with \textit{export OMP NUM THREADS=8}


\textbf{1. How many \textit{Hello world ...} lines are printed on the screen?}

\textbf{2. What does omp get num threads return when invoked outside and inside a parallel region?}

\subsection{data\_sharing.c}
\textbf{1. Which is the value of variable x after the execution of each parallel region with different datasharing
attribute (shared, private, firstprivate and reduction)? Is that the value you would expect? (Execute several times if necessary)}

\section{Loop parallelism}
\subsection{schedule.c}
\textbf{1. Which iterations of the loops are executed by each thread for each schedule kind?}

\subsection{2. nowait.c}
\textbf{1. Which could be a possible sequence of printf when executing the program?}
\textbf{2. How does the sequence of printf change if the nowait clause is removed from the first for
directive?}
\textbf{3. What would happen if dynamic is changed to static in the schedule in both loops? (keeping
the nowait clause)}
\subsection{collapse.c}
\textbf{1. Which iterations of the loop are executed by each thread when the collapse clause is used?}
\textbf{2. Is the execution correct if the collapse clause is removed? Which clause (different than
collapse) should be added to make it correct?}
\section{Synchronization}
\subsection{datarace.c} 
\textbf{1. Is the program always executing correctly?}
\textbf{2. Add two alternative directives to make it correct. Explain why they make the execution
correct.}
\subsection{barrier.c}
\textbf{1. Can you predict the sequence of messages in this program? Do threads exit from the barrier
in any specific order?}
\subsection{ordered.c}
\textbf{1.Can you explain the order in which the \textit{Outside} and \textit{Inside} messages are printed?}
\textbf{2. How can you ensure that a thread always executes two consecutive iterations in order during the execution of the ordered part of the loop body?}

\section{Tasks}
\subsection{1.single.c}
\textbf{1. Can you explain why all threads contribute to the execution of instances of the single worksharing
construct? Why are those instances appear to be executed in bursts?}
\subsection{fibtask.c}
\textbf{1. Why all tasks are created and executed by the same thread? In other words, why the program
is not executing in parallel?}
\textbf{2. Modify the code so that the program correctly executes in parallel, returning the same answer
that the sequential execution would return.}
\subsection{synchtasks.c}
\textbf{1. Draw the task dependence graph that is specified in this program}
\textbf{2. Rewrite the program using only taskwait as task synchronisation mechanism (no depend clauses allowed)}
\subsection{taskloop.c}
\textbf{1. Find out how many tasks and how many iterations each task execute when using the grainsize
and num tasks clause in a taskloop. You will probably have to execute the program several times in order to have a clear answer to this question.}
\textbf{2. What does occur if the nogroup clause in the first taskloop is uncommented?}

\section{Conclusion}




\end{document}