\documentclass[12]{article}
\usepackage[utf8]{inputenc}
\usepackage{cite}
\author{par4111 \\ Adrià Cabeza, Xavier Lacasa \\ Departament d' Arquitectura de Computadors}
\title{Lab 2: Brief tutorial on OpenMP programming model }
\date{\today \\ 2018 - 19 PRIMAVERA}
\usepackage{graphicx}
\usepackage{subcaption}
\usepackage{pgfplots}
\usepackage{listings}
\usepackage{color}
\usepackage{float}
\definecolor{mauve}{rgb}{0.58,0,0.82}
\definecolor{dkgreen}{rgb}{0,0.6,0}
\lstset{
	frame=tb,language=C,breaklines=true,numbers=none, commentstyle=\color{dkgreen}, stringstyle=\color{mauve}, tabsize=3,   showstringspaces=false,
  columns=flexible, 
}
\begin{document}
\maketitle

\newpage
\tableofcontents
\newpage
\section{Introduction}

In this session we will learn about the OpenMP programming model. 

\section{Parallel regions}
\subsection{hello.c}
\textbf{1. How many times will you see the "Hello world!" message if the program is executed with \textit{./1.hello}?}

We see the message 24 times. This is due to the \textit{\#pragma omp parallel} call before \textit{printf("Hello world! \textbackslash n");}, which makes every available thread execute the \textit{printf}. In Boada 1, they happen to be 24 threads, and so the message is printed 24 times.

\textbf{2. Without changing the program, how to make it to print 4 times the \textit{Hello World!} message?}

By setting the number of threads available to 4, only 4 threads would execute the \textit{printf("Hello world! \textbackslash n");} line and so the message would be displayed only 4 times. We can accomplish this by adding \textit{num\_threads(4)} after \textit{\#pragma omp parallel}.

\subsection{2.hello.c}
\textbf{1. Is the execution of the program correct? (i.e., prints a sequence of \textit{(Thid) Hello (Thid)
world!} being Thid the thread identifier). If not, add a data sharing clause to make it correct?} 

It is not correct, sometimes errors like \\
\textit{(2) Hello (1) Hello (1) world! (1) world!} \\
occur. These happen because the variable \textit{id} is declared before \textit{\#pragma omp parallel}, meaning all threads share it. Then what can happen is that for example thread 2 reads the variable \textit{id}, prints the \textit{"(2) Hello"} line, then thread 1 changes the variable, and when it is time to print the \textit{"(2) world!"}, thread 2 prints \textit{"(1) world!"} instead, because thread 1 assigned a new value before thread 2 was done.
\\
To make it correct, we can add the \textit{private(id)} tag to the \textit{\#pragma omp parallel num\_threads(8)} line, so that every thread has its own local value of the variable \textit{id}. This way, when they assign their own id, they do not change the value other threads are reading from the var \textit{id}.

\textbf{2. Are the lines always printed in the same order? Why the messages sometimes appear intermixed?
(Execute several times in order to see this).}
No, the lines are not always printed in the same order. This happens because threads do not get tasks assigned by \#id order, so whichever thread gets the task earlier in that execution will print earlier.
\\
Messages appear intermixed because even though we added the \textit{private(id)} tag, threads do not wait for other threads to finish their execution before printing their lines (that would be sequential instead of parallel). This way, one thread might print \textit{(id) Hello} and right then another thread's execution might start, printing \textit{(id) Hello} again with a different \textit{id}. After that, they both will finish with \textit{(id) world!} and messages will have been intermixed. To avoid this, we should make the execution of the other threads stop when one thread starts to print until it finishes (we could use the \textit{critical} construct), but that would make the execution sequential (with the added overhead of parallelization, even worse).

\subsection{how\_many.c}
Assuming the \textit{OMP NUM THREADS} variable is set to 8 with \textit{export OMP NUM THREADS=8}


\textbf{1. How many \textit{Hello world ...} lines are printed on the screen?}

\textbf{2. What does omp get num threads return when invoked outside and inside a parallel region?}

\subsection{data\_sharing.c}
\textbf{1. Which is the value of variable x after the execution of each parallel region with different datasharing
attribute (shared, private, firstprivate and reduction)? Is that the value you would expect? (Execute several times if necessary)}

\section{Loop parallelism}
\subsection{schedule.c}
\textbf{1. Which iterations of the loops are executed by each thread for each schedule kind?}

\subsection{2. nowait.c}
\textbf{1. Which could be a possible sequence of printf when executing the program?}
\textbf{2. How does the sequence of printf change if the nowait clause is removed from the first for
directive?}
\textbf{3. What would happen if dynamic is changed to static in the schedule in both loops? (keeping
the nowait clause)}
\subsection{collapse.c}
\textbf{1. Which iterations of the loop are executed by each thread when the collapse clause is used?}
\textbf{2. Is the execution correct if the collapse clause is removed? Which clause (different than
collapse) should be added to make it correct?}
\section{Synchronization}
\subsection{datarace.c} 
\textbf{1. Is the program always executing correctly?}
\textbf{2. Add two alternative directives to make it correct. Explain why they make the execution
correct.}
\subsection{barrier.c}
\textbf{1. Can you predict the sequence of messages in this program? Do threads exit from the barrier
in any specific order?}
\subsection{ordered.c}
\textbf{1.Can you explain the order in which the \textit{Outside} and \textit{Inside} messages are printed?}
\textbf{2. How can you ensure that a thread always executes two consecutive iterations in order during the execution of the ordered part of the loop body?}

\section{Tasks}
\subsection{1.single.c}
\textbf{1. Can you explain why all threads contribute to the execution of instances of the single worksharing
construct? Why are those instances appear to be executed in bursts?}
\subsection{fibtask.c}
\textbf{1. Why all tasks are created and executed by the same thread? In other words, why the program
is not executing in parallel?}
\textbf{2. Modify the code so that the program correctly executes in parallel, returning the same answer
that the sequential execution would return.}
\subsection{synchtasks.c}
\textbf{1. Draw the task dependence graph that is specified in this program}
\textbf{2. Rewrite the program using only taskwait as task synchronisation mechanism (no depend clauses allowed)}
\subsection{taskloop.c}
\textbf{1. Find out how many tasks and how many iterations each task execute when using the grainsize
and num tasks clause in a taskloop. You will probably have to execute the program several times in order to have a clear answer to this question.}
\textbf{2. What does occur if the nogroup clause in the first taskloop is uncommented?}

\section{Conclusion}




\end{document}